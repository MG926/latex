\documentclass[12pt, a4paper, oneside]{ctexart}
\usepackage{amsmath, amsthm, amssymb, graphicx,listings,xcolor}
\usepackage{amsfonts}

\usepackage[bookmarks=true, colorlinks, citecolor=blue, linkcolor=black]{hyperref}
\usepackage{afterpage}
\newcommand\myemptypage{
    \null
    \thispagestyle{empty}
    \addtocounter{page}{-1}
    \newpage
    }

\usepackage{geometry}
\usepackage{setspace}
\geometry{left=3.18cm,right=3.18cm,top=2.54cm,bottom=2.54cm}

\pagestyle{plain}	
% \usepackage{booktabs}
% \usepackage{subfigure}

\date{}
\begin{document}
    
	\begin{figure}[t]
		\parbox[b]{2cm}{
			\includegraphics[scale=0.2]{校徽.jpg}
			}
            \parbox[b]{9cm}{
                \begin{center}
                    

				\small \textbf{\quad \quad \quad SOUTHWEST JIAOTONG\quad UNIVERSITY} 
			\end{center}
			}
	\end{figure}

	\begin{center}
		\quad \\
		\quad \\
		\heiti \fontsize{45}{17} 高\quad 数\quad 作\quad 业
		\vskip 3.5cm
		\heiti \zihao{2} 知识点总结	
	\end{center}
	\vskip 3.5cm

	\begin{quotation}
		\songti \fontsize{15}{15}
		\doublespacing
		\par\setlength\parindent{12em}
		\quad 
        
	    学\hspace{0.61cm} 院:\underline{电气工程学院\quad}
        
	    专\hspace{0.61cm} 业:\underline{电子信息工程}
        
	    学生姓名:\underline{\qquad MG\qquad }
        
	    学\hspace{0.61cm} 号:\underline{\quad 2024111878\quad}

	    指导教师:\underline{\qquad 陈桂玲 \qquad}
		\vskip 2cm
		\centering
		2023年12月24日
		
        \centering
        \LaTeX
	\end{quotation}

	\newpage
	\myemptypage

    \tableofcontents

    \newpage
	\myemptypage

\section{摘要}
临近期末,陈桂林老师发布高数论文————知识点总结(第四章至第七章前三节)期末大作业。学生认为数学的解题思路是活的,对应不同题目有不同思路,同一道题目亦有不同思路,然而针对题目采取知识点的总结会导致本论文内容过多,故本论文决定采取对书上的定义定理以及一些方法进行汇总,方便期末复习时对定义定理进一步加深印象。最后决定采用\LaTeX 的形式完成论文也是对\LaTeX 写作能力的一次锻炼。
\vskip 2cm

\section{不定积分}
\vskip 2cm
\subsection{不定积分的概念与性质}
\vskip 1cm
\subsubsection{原函数与不定积分的概念}
\paragraph{定理4.1 (原函数存在定理)} 若函数f(x)在区间I上连续,则在区间I上存在可导函数F(x),使得对任意一x$ \in $I都有F'(x)=f(x).
\newline
简单来说:\textbf{连续函数一定存在原函数}.

\paragraph{定理4.2 (原函数之间的关系定理)} 如果函数f(x)在区间I上存在原函数,那么其任意两个原函数之间只相差一个常数项.

\paragraph{定义4.2 (不定积分的定义)} 函数f(x)在区间I上的全体原函数所形成的函数族称为f(x)在区间I上的\textbf{不定积分} ,记作 $\int $ f(x)dx.其中记号“$\int $”称为\textbf{积分号},f(x)称为\textbf{被积函数},f(x)dx称为\textbf{被积表达式} ,x称为\textbf{积分变量} ,C称为\textbf{积分常数}.

\vskip 2cm
\subsubsection{不定积分的性质}
\vskip 1cm
\paragraph{定理4.3 (不定积分的定义)}
\begin{enumerate}
    \item[1.]微分运算与积分运算互为逆运算. \begin{enumerate}
        \item [(1)][$\int $f(x)dx]'=f(s) 或d[$\int $f(x)dx]=f(x)dx.
        \item [(2)]$\int $F'(x)dx=F(x)+C或$\int $dF(x)=F(x)+C;特别地,有$\int $dx=x+C.
    \end{enumerate}
    \item[2.]被积函数中不为零的常数因子可以移到积分号的前面,即
    \begin{center}
        $\int $ kf(x)dx=k$\int $f(x)dx(k为非零常数).
    \end{center}
    \item[3.]两个函数的和(或差)的不定积分等于各个函数的不定积分的和(或差),即
    \begin{center}
        $\int $ [f(x)±g(x)]dx=$\int $f(x)dx + $\int $g(x)dx.
    \end{center}

\end{enumerate}



\subsection{换元积分法}
\vskip 1cm
\subsubsection{第一换元积分法}
\paragraph{定理4.4(第一类换元积分法)}设$\int $f(u)du=F(u)+C. 如果u=$\varphi $具有连续导数,那么
\begin{center}
    $\int $f($\varphi $(x))$\varphi $'(x)dx=F($\varphi $(x))+C.
\end{center}

\vskip 1cm
\subsubsection{第二换元积分法}
\paragraph{定理4.5(第二类换元积分法)}设y=f(x),f=$\varphi $(t)及x'=$\varphi $'(t)均为连续函数,且$\varphi $'(t)$\neq $0,并设t=$\varphi _{}^{-1}$(x)为x=$\varphi $(t)的反函数,若F(t)是f($\varphi $(t))$\varphi $'(t)的一个原函数,则
\begin{center}
    $\int $f(x)dx=$\int $f($\varphi $(t))$\varphi $'(t)dt=F(t)+C=F($\varphi _{}^{-1}$(t))+C.

\end{center}

\vskip 1cm
\subsection{分部积分法}
\vskip 1cm
\paragraph{定理4.6(分部积分公式)}设函数u=u(x),v=v(x)的导函数连续,则
\begin{center}
    $\int $udv=uv-$\int $vdu或者$\int $uv'dx=uv-$\int $vu'dx
\end{center}

\vskip 1cm
\subsection{有理函数的不定积分}
\vskip 1cm
\subsubsection{有理函数的不定积分}
\paragraph{定义4.3(有理函数的定义)}形如
\begin{center}
    $\frac{P(x)}{Q(x)}$=$\frac{a_{0}^{} x_{}^{m} +a_{1}^{} x_{}^{m-1} +\dots +a_{m}^{} }{b_{0}^{} x_{}^{n} +b_{1}^{} x_{}^{n-1} +\dots +b_{n}^{} }$
\end{center}
(其中n,m为正整数,$a_{0}^{}$ ,$a_{1}^{}$ ,\dots ,$a_{m}^{}$ 及$b_{0}^{} $,$b_{1}^{}$ ,\dots ,$b_{n}^{} $为实常数,且$a_{0}^{} $≠0,$b_{0}^{} $≠0)
\newline
的函数称为\textbf{有理函数}(也称为\textbf{有理分式}).当n>m时,称$\frac{P(x)}{Q(x)}$ 为\textbf{真分式};当n$\leq $m时,称$\frac{P(x)}{Q(x)}$ 为\textbf{假分式}.
		
\paragraph{推论}对于真分式$\frac{P(x)}{Q(x)}$可以展开成如下的求和表达式:

\begin{equation}
	\frac{P(x)}{Q(x)}=\sum_{j=1}^{l} \Big(\sum_{k=1}^{k_{j}} \frac{a_{jk}^{}}{(x-x_{j}^{})_{}^{k}} \Big)+\sum_{j=1}^{n} \Big(\sum_{k=1}^{m_{j}} \frac{b_{jk}^{}x+c_{jk}^{}}{({x_{}^{2}+p_{j}^{}x+q_{j}^{}})^{k}} \Big)
\end{equation}
其中$a_{jk}^{}$,$b_{jk}^{}$,$c_{jk}^{}$是唯一确定的实数,而
\begin{equation}
	Q(x)=(x-x_{1}^{})_{}^{k1}\dots (x-x_{l}^{})_{}^{kl}(x_{}^{2}-p_{1}^{}x+q_{1}^{})_{}^{m1}\dots (x_{}^{2}-p_{n}^{}x+q_{n}^{})_{}^{m_{n}^{}}
\end{equation}

\newpage
\section{定积分}
\vskip 2cm
\subsection{定积分的概念与性质}
\vskip 1cm
\subsubsection{定积分的概念}
\paragraph{定理5.1(函数可积的充分条件)}
\begin{enumerate}
	\item [1.]若函数f(x)在闭区间[a,b]上连续,则f(x)在[a,b]上可积.
	\item [2.]若函数f(x)在闭区间[a,b]上除去有限多个间断点外处处连续,且在[a,b]上有界,则f(x)在[a,b]上可积.
	\item [3.]若函数f(x)在闭区间[a,b]上有界且单调,则f(x)在[a,b]上可积.
\end{enumerate}


\paragraph{定理5.2(定积分的性质)}
\begin{enumerate}
	\item [1.]\textbf{几何度量性质:}如果在区间[a,b]上f(x)$\equiv $1,那么
	\begin{equation}
		\int_{a}^{b}1dx=\int_{a}^{b}f(x)dx=b-a.
	\end{equation}
	\item [2.]\textbf{线性性质:}设$k_{1}^{}$,$k_{2}^{}$均为常数,则
	\begin{equation}
		\int_{a}^{b}[k_{1}^{}f(x)+k_{2}^{}g(x)]dx=k_{1}^{}\int_{a}^{b}f(x)dx+k_{2}^{}\int_{a}^{b}g(x)dx
	\end{equation}
	\item [3.]\textbf{积分区间的可加性.}
	\item [4.]\textbf{保号性:}如果在区间[a,b]上恒有f(x)$\geq$ 0,那么$\int _{a}^{b}f(x)dx\geq $0.
	\item [5.]\textbf{估值定理}设M和m分别是函数f(x)在区间[a,b]上的最大值和最小值,则
	\begin{equation}
		m(b-a)\leq \int _{a}^{b}f(x)dx\leq M(b-a).
	\end{equation}
	\item [6.]\textbf{积分中值定理:}如果函数f(x)在区间[a,b]上连续,那么在[a,b]上至少存在一点$\xi$,使得下式成立:
	\begin{equation}
		\int _{a}^{b}f(x)dx=f(\xi)(b-a).
	\end{equation}
\end{enumerate}

\paragraph{推论1(单调性)}如果在区间[a,b]上恒有f(x)$\leq$g(x),那么
\begin{equation}
	\int_{a}^{b}f(x)dx\leq \int _{a}^{b}g(x)dx.
\end{equation} 

\paragraph{推论2(绝对值不等式)}
\begin{equation}
	\left|\int_{a}^{b}f(x)dx\right|\leq \int _{a}^{b}\left|f(x)\right|dx(a \textless b).
\end{equation} 

\vskip 1cm
\subsection{微积分基本公式}
\subsubsection{积分上限函数及其导数}
\paragraph{定理5.3(积分上限函数的导数公式)}若函数f(x)在区间[a,b]上连续,则积分上限函数$\phi$(x)=$\int_{a}^{x}$f(t)dt在[a,b]上可导,并且它的导数
\begin{equation}
	\phi'(x)=\frac{d}{dx}\int _{a}^{x}f(t)dt=f(x)(a\leq x\leq b).
\end{equation}

\subsubsection{微积分基本定理}
\paragraph{定理5.4(微积分基本定理)}设函数f(x)在区间[a,b]上连续,若函数F(x)是f(x)在区间[a,b]上的一个原函数,则
\begin{equation}
	\int_{a}^{b}f(x)dx=F(b)-F(a).
\end{equation}

\vskip 2cm
\subsection{定积分的换元积分法和分部积分法}
\vskip 1cm
\subsubsection{定积分的换元积分法}
\paragraph{定理5.7(定积分的换元积分法)}设函数y=f(x)在区间[a,b]上连续,函数x=$\phi $(t)满足下列条件:
\begin{enumerate}
	\item[(1)]$\phi $($\alpha$)=a,$\phi $($\beta $)=b;
	\item[(2)]x=$\phi$(t)在[$\alpha $,$\beta $](或[$\beta $,$\alpha $])上具有连续导数,且其值域$R_{\phi }^{}$=[a,b],则有
	\begin{equation}
		\int_{a}^{b}f(x)dx=\int_{\alpha}^{\beta}f(\phi (t))\phi '(t)dt.(\textbf{换元积分公式})
	\end{equation} 
\end{enumerate}
\paragraph{注:}“换元必换限”

\subsubsection{定积分的分部积分法}
\begin{equation}
	\int _{a}^{b}u'vdx=[uv]_{a}^{b}-\int _{a}^{b}uv'dx.
\end{equation}
\paragraph{注:}“边积边代限”

\vskip 1cm
\subsection{反常积分}
\subsubsection{无穷区间上的反常积分}
\paragraph{定义5.2(无穷区间上的反常积分的敛散性)}无穷区间\textbf{连续},存在积分极限,则反常积分\textbf{收敛},否则\textbf{发散}.
\subsubsection{无界区间上的反常积分}
\paragraph{瑕点}如果函数y=f(x)在点a的任一邻域内都无界,则点a称为y=f(x)的\textbf{瑕点(奇点)},也称为\textbf{无穷间断点}.
\paragraph{定义5.3(无界函数的反常积分的敛散性)}对于无界函数,在其瑕点邻域内,若存在积分极限,则反常积分收敛,否则发散.
\subsubsection{反常积分敛散性的判别法}
\paragraph{定理5.8(无穷区间上的反常积分的直接比较判别法)}设函数f和g在区间[a,+$\infty$)上连续,且对所有的x$\geq $a,有0$\leq $f(x)$leq $g(x).
\begin{enumerate}
	\item [1.]若$\int _{a}^{+\infty}$g(x)dx收敛,则$\int _{a}^{+\infty }$f(x)dx收敛.
	\item [2.]若$\int _{a}^{+\infty}$f(x)dx发散,则$\int _{a}^{+\infty }$g(x)dx发散.
\end{enumerate}

\subsubsection{反常积分的Cauchy主值}
对于在瑕点邻域不可积分的无界函数,可以对其积分值赋予Cauchy主值的定义,用符号\textbf{V.P.“Valeur principale"}表示.即,在这种情况下,通常说\textbf{反常积分在主值的意义下存在}.

\section{定积分的应用}
\subsection{几个概念}
\begin{enumerate}
	\item [1.]面积.
	\item [2.]体积.
	\item [3.]弧长.
	\item [4.]曲率.
\end{enumerate}
\paragraph{注:}具体概念不展开论述.
\subsection{几种方法}
\begin{enumerate}
	\item [1.]定积分的微元法(定义法).
	\item [2.]求面积的矩形法,求面积的柱壳法、截面法.
	\item [3.]不同坐标系下的积分(直角坐标,极坐标,复平面).
\end{enumerate}
\paragraph{注:}具体方法不展开论述.


\newpage
\section{微分方程}
\subsection{微分方程的几种类型}
\paragraph{变量可分离微分方程}对于以下形式的微分方程:
\begin{equation}
	y'=f(x,y) \quad \text {或} \quad \frac{dy}{dx}=f(x,y).
\end{equation}
可以写成以下形式:
\begin{equation}
	P(x,y)dx+Q(x,y)dy=0.
\end{equation}
\vskip 1cm
\paragraph{齐次方程}若一节微分方程可化为:
\begin{equation}
	\frac{dy}{dx}=\phi(\frac{y}{x}).
\end{equation}
的形式,则这种方程为\textbf{齐次方程}.
\vskip 1cm
\paragraph{伯努利方程}方程
\begin{equation}
	\frac{dy}{dx}+P(x)y=Q(x)y'',(n\neq 0,1)
\end{equation}
称为\textbf{伯努利(Bernoulli)方程}.

\subsection{解微分方程的几种方法}
\paragraph{两边对称积分法}对于形如f(x)dx=g(y)dy的微分方程,可以两边同时进行不定积分,得到该微分方程的解.
\paragraph{一阶线性微分方程的公式法}对于形如y'+P(x)y=Q(x)的一阶线性微分方程,有如下通解:
\begin{equation}
	y=Ce_{}^{-\int P(x)dx}+e_{}^{-\int P(x)dx}\int Q(x)e_{}^{\int P(x)dx}dx.
\end{equation}
\paragraph{注:}推导过程(\textbf{常数变易法})在此不做阐述.

\newpage
\section{结语}
数山有路勤为径,学海无涯苦作舟.







\end{document}